\thesistitle{Computing as Context: Experiences of Dis/Connection Beyond the Moment of Non/Use}

\degreename{Doctor of Philosophy}

% Use the wording given in the official list of degrees awarded by UCI:
% http://www.rgs.uci.edu/grad/academic/degrees_offered.htm
\degreefield{Information and Computer Science}

% Your name as it appears on official UCI records.
\authorname{Mary E. Harmon}

% Use the full name of each committee member.
\committeechair{Associate Professor Melissa A. Mazmanian}
\othercommitteemembers
{
  Associate Professor Kavita Sara Philip\\
  Professor J. Paul Dourish\\
  Professor Geoffrey C. Bowker
}

\degreeyear{2015}

\copyrightdeclaration
{
  © {\Degreeyear} \Authorname
}

% If you have previously published parts of your manuscript, you must list the
% copyright holders; see Section 3.2 of the UCI Thesis and Dissertation Manual.
% Otherwise, this section may be omitted.
% \prepublishedcopyrightdeclaration
% {
% 	Chapter 4 {\copyright} 2003 Springer-Verlag \\
% 	Portion of Chapter 5 {\copyright} 1999 John Wiley \& Sons, Inc. \\
% 	All other materials {\copyright} {\Degreeyear} \Authorname
% }

% % The dedication page is optional.
% \dedications
% {
%   (Optional dedication page)
%
%   To ...
% }

\acknowledgments
{
The research for this dissertation has been funded in part by the Intel Science and Technology Center for Social Computing, a Rob Kling Memorial Fellowship, and a Dean's Fellowship from the UCI School of Information and Computer Science. 
	
This dissertation gives but a few small glimpses into the lives of my many fieldwork participants. I am indebted to all of them for numerous insightful interviews, casual conversations, and welcoming me into their homes, workplaces, backyards, campsites, and lives.
  
In addition to my advisor, Melissa Mazmanian, and committee members Kavita Philip, Paul Dourish, and Geof Bowker, I would also like to thank Gary Olson, Judy Olson, and Bill Maurer who worked with me at UCI prior to the dissertation, and Christine Beckman who has been an invaluable collaborator in the research conducted with families and workers in southern California.

This dissertation would not have been possible without the support, encouragement, and intellectual engagements of my family, friends, and colleagues. In particular, I would like to thank: Luke Olbrish, Sue Harmon, Lynn Dombrowski, Kate Darling, Mel Gregg, Lilly Irani, Nick Seaver, Six Silberman, Dillon Mahmoudi, Erin Goodling, Anthony Levenda, Oliver Haimson, Beth Reddy, Luke Stark, Austin Toombs, Sen Hirano, Katie Pine, Marisa Cohn, Jed Brubaker, Amy Voida, and Nancy Nersessian. 
 }


% Some custom commands for your list of publications and software.
\newcommand{\mypubentry}[4]{
    \textbf{#1} \hfill \textbf{#2} \\ 
    #3 \\
		#4
}

% Include, at minimum, a listing of your degrees and educational
% achievements with dates and the school where the degrees were
% earned. This should include the degree currently being
% attained. Other than that it's mostly up to you what to include here
% and how to format it, below is just an example.
\curriculumvitae
{

\textbf{EDUCATION}
 
    \textbf{Doctor of Philosophy in Information and Computer Science} \hfill \textbf{2015}\\
    University of California, Irvine
  
    \textbf{Master of Science in Human Computer Interaction} \hfill \textbf{2007} \\
    \textbf{Bachelor of Science in Computer Science} \hfill \textbf{2004} \\
    Georgia Institute of Technology\\

\textbf{TEACHING EXPERIENCE}

    \textbf{Instructor} \hfill \textbf{2014} \\
    Introduction to Human Computer Interaction \\
    Department of Informatics, University of California, Irvine\\

	
\textbf{SELECTED PUBLICATIONS}

  \mypubentry{Stories of the smartphone in everyday discourse: conflict, tension \& instability}{2013}{Ellie Harmon and Melissa Mazmanian}{\emph{ACM Conference on Human Factors in Computing Systems (CHI '13)}}
	
  \mypubentry{Edit-work: promoting interdisciplinary conversation within flagship journals}{2012}{Ellie Harmon}{In Chima M. Anyadike-Danes et al., ``Reflections on american anthropology: \\
	a conference at UC Irvine.''\emph{ American Anthropologist.}}
	
  \mypubentry{Cognitive partnerships on the bench top: designing to support scientific researchers}{2008}{Ellie Harmon and Nancy J. Nersessian}{\emph{ACM Conference on Designing Interactive Systems (DIS '08)}}\\
	
	\textbf{GALLERY EXHIBITS}
	
		\mypubentry{DIRT}{2014}{Christina Agapakis and Ellie Harmon}{\emph{UCLA Art|Science Center}, May 1 - 15}

}

% The abstract should not be over 350 words, although that's
% supposedly somewhat of a soft constraint.
\thesisabstract
{
What does it mean to be “constantly connected” or to work for a “24/7” company? What does it mean to ``disconnect'' in an era of “always on” connectivity? This dissertation examines some of the textures of American life in an historical moment marked both by the arrival of ubiquitous computing and the development of a broad-based conversation about the value and merits of ‘disconnection.’

Taking a multi-sited ethnographic approach, I trace ``connection'' and ``disconnection'' as they manifest in discourse, practice, and lived experience across multiple scenes of American life -- the suburban household, the contemporary workplace, disconnection retreats, and the wilderness of the Pacific Crest Trail.

One of the central empirical findings of this dissertation is that connectivity was not \emph{constant} for the participants in this research. Rather, observed patterns of technology use were \emph{punctuated} and \emph{variegated}. Yet, many of these same participants also often described their lives as ``constantly connected'' and expressed desires to disconnect. 
The dissertation thus argues for the importance of separating analytically the diffuse and pervasive experiences of computing from moments of ‘interaction’ and ‘use’ that have traditionally been the focus Informatics and related fields.

Instead, I argue for attending to \emph{computing as the context} for social life and action, a perspective that emphasizes the ways that people situationally \emph{arrange} computing artifacts and leverage constraints to shape their social life. This perspective also suggests a new understanding of \emph{disconnection} as something less about unplugging from technological objects, and more about \emph{a context shift, and social reconfiguration}. When people disconnect, they are also altering possibilities for social interaction, and concommitant expectations and obligations. That is, disconnection appears as a \emph{proxy} for short-circuiting habits and patterns of social life that exceed moments of device interaction and tool use.

More broadly, this research suggests that there are significant limits to the possibilities of design to intervene in contemporary scenes of busyness and overwhelm, and draws attention to the ways that social values are produced through computing use and adoption, rather than perfectly embedded within technological artifacts through the intentions of the designer.
}


%%% Local Variables: ***
%%% mode: latex ***
%%% TeX-master: "thesis.tex" ***
%%% End: ***
